\documentclass[memoire.tex]{subfiles}

\chapter{Présentation}

Aujourd'hui le Big Data est devenu indispensable pour les entreprises. Malgrès les barrières présentes pour son implémentation, le Big Data permet aux entreprises de s'améliorer grandement dans leurs activités, notamment avec l'utilisation d'intelligence ou bien le machine learning.~\cite{1}
Le point commun de tout les outils concernant le Big Data est le traitement des données, c'est un point très important car il faut bien le choisir en fonction des besoins pour eviter un surcoût et donc une perte d'efficacité. Il existe trois types de traitement de données dans le Big Data : 
\begin{itemize}
\item Batch
\item Micro Batch
\item Streaming
\end{itemize}

A travers ce mémoire, je vais parcourir ces trois types de traitements de données ainsi que les solutions déjà existante avant de pouvoir réaliser un tableau comparatif pour définir si une solution est meilleure que les autres ou bien pour choisir la bonne solution selon les beoins.
\cite{2}
\cite{3}
\cite{4}
\cite{5}

Dans un second temps, je vais essayer de proposer une nouvelle solution pour le type de traitement le plus adequat par rapport aux comparaisons réalisés en amont.