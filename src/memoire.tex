\documentclass[12pt, twoside, openright]{report}

%----------------------------------------------------------------------------------------
%	PACKAGES
%----------------------------------------------------------------------------------------

\usepackage{emptypage}
\usepackage{geometry}
\usepackage[utf8x]{inputenc}
\usepackage[french]{babel}
\usepackage[T1]{fontenc}
\usepackage{amsmath}
\usepackage{amsfonts}
\usepackage{amssymb}
\usepackage{graphicx}
\usepackage{subfiles}
\usepackage{fullpage}
\usepackage{fancyhdr}
\usepackage{shorttoc}
\usepackage{fontspec}
\usepackage{xltxtra}
\usepackage{xcolor}
\usepackage{sectsty}
\usepackage[Lenny]{fncychap}
\usepackage[backend=bibtex,sorting=none]{biblatex}
\usepackage[titletoc]{appendix}

%----------------------------------------------------------------------------------------
%	BIBLIOGRAPHIE
%----------------------------------------------------------------------------------------

\addbibresource{bibliographie/bibliographie.bib}

%----------------------------------------------------------------------------------------
%	STYLES
%----------------------------------------------------------------------------------------

\ChNumVar{\fontsize{60}{62}\usefont{OT1}{ptm}{m}{n}\selectfont\textcolor{red}}

\setmainfont[Mapping=tex-text]{Lato}

\geometry{
    paper=a4paper,
    inner=3cm,
    outer=2.5cm,
    top=2.5cm,
    bottom=3.5cm
}

\pagestyle{fancy}
\usepackage{etoolbox}
\patchcmd{\chapter}{\thispagestyle{plain}}{\thispagestyle{fancy}}{}{}
\renewcommand\headrulewidth{0pt}
\fancyhead[L]{}
\fancyhead[C]{}
\fancyhead[R]{}

\makeatletter
\patchcmd{\@makechapterhead}{\vspace*{50\p@}}{\vspace*{-35\p@}}{}{}
\patchcmd{\@makeschapterhead}{\vspace*{50\p@}}{\vspace*{-35\p@}}{}{}
\patchcmd{\DOTI}{\vskip 80\p@}{\vskip 40\p@}{}{}
\patchcmd{\DOTIS}{\vskip 40\p@}{\vskip 0\p@}{}{}
\makeatother

\renewcommand\thesection{\color{red}\thechapter.\arabic{section}}

\newenvironment{acknowledgements} {\renewcommand\abstractname{Remerciements}\begin{abstract}} {\end{abstract}}


\setcounter{tocdepth}{3}
\setcounter{secnumdepth}{3}

\newcommand*{\captionsource}[2]{%
  \caption[{#1}]{%
    #1%
    \\\hspace{\linewidth}%
    \textbf{Source:} #2%
  }%
}

%----------------------------------------------------------------------------------------
%	INFORMATIONS
%----------------------------------------------------------------------------------------

\author{Ludwig SIMON}
\title{Mémoire de fin d'études}

\begin{document}

%\subfile{cover/page_garde}
%----------------------------------------------------------------------------------------
%	TITLE PAGE
%----------------------------------------------------------------------------------------

\setlength{\parindent}{0cm}
\setlength{\parskip}{1ex plus 0.5ex minus 0.2ex}
\newcommand{\hsp}{\hspace{20pt}}
\newcommand{\HRule}{\rule{\linewidth}{0.5mm}}

\begin{titlepage}
  \begin{sffamily}
  \begin{center}

    \textsc{\LARGE MASTER MIAGE 2ème année \linebreak Université Paris Nanterre}\\[2cm]

    \textsc{\Large Mémoire de fin d'études}\\[1.5cm]

    \HRule \\[0.4cm]
    { \huge \bfseries Méthodes d'analyse des processus métier pour le choix d'une architecture Big Data adaptée\\[0.4cm] }

    \HRule \\[2cm]
    \includegraphics[scale=0.40]{img/logo_nanterre.jpg}
    \hspace{2cm}
    
    \vfill
  \begin{minipage}{0.4\textwidth}
      \begin{flushleft} \large
        \emph{Auteur :}\\ \textsc{Ludwig SIMON}\\
      \end{flushleft}
    \end{minipage}
    \begin{minipage}{0.4\textwidth}
      \begin{flushright} \large
        \emph{Tuteur :}\\ \textsc{Mcf. Emmanuel HYON}\\
      \end{flushright}
    \end{minipage}
    \vfill
    {\large Février 2019 — Juin 2019}
  \end{center}
  \end{sffamily}
\end{titlepage}

\leavevmode\thispagestyle{empty}\newpage

%----------------------------------------------------------------------------------------
%	Remerciement
%----------------------------------------------------------------------------------------

\begin{acknowledgements}

Je tiens tout d'abord à remercier mon tuteur, monsieur Emmanuel Hyon qui m'a suivi toute l'année pendant la rédaction de mémoire. Les conseils et l'aide qu'il a pu m'apporter sur la rédaction de ce dernier m'ont vraiment été très utiles.

J'aimerais aussi remercier mes collègues d'EDF avec qui j'ai pu travailler dans de bonnes conditions tout au long de l'année. Je les remercie aussi pour le temps qu'ils m'ont accordé afin de répondre à mes interrogations sur le domaine du Big Data pour m'aider dans la rédaction de ce mémoire.

\end{acknowledgements}

\leavevmode\thispagestyle{empty}\newpage

%----------------------------------------------------------------------------------------
%	RESUME
%----------------------------------------------------------------------------------------

\begin{abstract}

Le Big Data est de plus en plus présent aujourd'hui, et pourtant ce sujet reste trop vague pour un trop grand nombre de personnes qui se laisse aveugler par la popularité d'une technologie pour concevoir leur architecture Big Data.

De plus certaines entreprises tentent le Big Data sans avoir de cas utilisation concret, cela engendre une architecture complexe et chère à maintenir alors qu'ils n’en ont nullement l'utilité.

Dans un premier temps, nous allons brièvement rappeler ce qu'est le Big Data et montrer à partir de quand il est intéressant de l'utiliser.

Dans un second temps, nous allons faire un état des lieux des composants qui composent une architecture Big Data et après nous allons nous intéresser aux différents modèles d'architectures Big Data existants.

Ensuite, nous allons nous parcourir les différentes solutions logicielles intervenant dans les architectures Big Data que nous avons étudiées. Cela permettra d'analyser leur fonctionnement afin de trouver des critères de sélection.

Dans une troisième partie, nous établirons les critères permettant de départager les différentes architectures Big Data ainsi que les différentes solutions logicielles permettant la réalisation de celles-ci.

Finalement, nous définirons une manière d'évaluer notre choix d'architecture afin de vérifier que c'est bien la solution optimale.

\end{abstract}

\leavevmode\thispagestyle{empty}\newpage

%----------------------------------------------------------------------------------------
%	Pre Face
%----------------------------------------------------------------------------------------

\section*{Motivations}
Avant d'avoir commencé mon stage de Master 1 chez EDF dans le service Big Data, pour moi le Big Data était une notion très floue et je ne savais pas vraiment tout ce qui composait ce domaine. Durant ce stage j'ai pu découvrir ce domaine et me rendre compte du nombre conséquent d'outils différents qui sont nécessaires pour la conception d'une architecture Big Data complète. C'est à partir de ce moment-là que je me suis demandé, comment est-ce que le choix de l'architecture, et des outils Big Data s'effectue ? Il y a beaucoup plus de possibilités que dans les domaines que j'ai pu voir jusqu'à maintenant. C'est pour cela que j'ai décidé cette année, d'essayer de comprendre comment fonctionne une architecture Big Data afin de trouver une solution pour choisir plus facilement une architecture Big Data adaptée à nos besoins. 

\section*{Objectifs}

À travers ce mémoire j'aimerais dans un premier temps vous rappeler ce que c'est que le Big Data. Le Big Data étant devenu à "la mode", je pense qu'il est important de rappeler ce que c'est exactement et quand il est intéressant de l'utiliser.

Dans un second temps, je voudrais vous montrer à quel ce domaine est vaste et à quel point c'est important de se renseigner de manière régulière sur les nouveautés afin de ne pas faire un choix basé uniquement sur la popularité d'un outil. Il est important de faire soit même ses recherches sur les différents outils à notre disposition afin d'utiliser celui qui correspond le mieux à nos utilisations.

Et pour finir, j'aimerais définir des critères permettant à partir d'un cas d'utilisation, de nous proposer la meilleure architecture Big Data.


\leavevmode\thispagestyle{empty}\newpage

%----------------------------------------------------------------------------------------
%	SOMMAIRE
%----------------------------------------------------------------------------------------

\shorttoc{Sommaire}{1}

%----------------------------------------------------------------------------------------
%	INTRODUCTION
%----------------------------------------------------------------------------------------

\subfile{introduction}

%----------------------------------------------------------------------------------------
%	CHAPITRES
%----------------------------------------------------------------------------------------
\subfile{chapitres/chapitre1}

\subfile{chapitres/chapitre2}

\subfile{chapitres/chapitre3}

\subfile{chapitres/chapitre4}

%----------------------------------------------------------------------------------------
%	CONCLUSION
%----------------------------------------------------------------------------------------

\subfile{conclusion}

%----------------------------------------------------------------------------------------
%	ANNEXES
%----------------------------------------------------------------------------------------

\subfile{annexes/annexes}

%----------------------------------------------------------------------------------------
%	BIBLIOGRAPHIE
%----------------------------------------------------------------------------------------

\printbibliography[heading=bibintoc]
\newpage

%----------------------------------------------------------------------------------------
%	TABLE DES MATIERES
%----------------------------------------------------------------------------------------

\tableofcontents

\listoffigures

\listoftables

\end{document}
