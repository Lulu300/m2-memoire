\documentclass[12pt, twoside, openright]{report}

%----------------------------------------------------------------------------------------
%	PACKAGES
%----------------------------------------------------------------------------------------

\usepackage{emptypage}
\usepackage{geometry}
\usepackage[utf8x]{inputenc}
\usepackage[french]{babel}
\usepackage[T1]{fontenc}
\usepackage{amsmath}
\usepackage{amsfonts}
\usepackage{amssymb}
\usepackage{graphicx}
\usepackage{subfiles}
\usepackage{fullpage}
\usepackage{fancyhdr}
\usepackage{shorttoc}
\usepackage{fontspec}
\usepackage{xltxtra}
\usepackage{xcolor}
\usepackage{sectsty}
\usepackage[Lenny]{fncychap}
\usepackage[backend=bibtex,sorting=none]{biblatex}
\usepackage[titletoc]{appendix}

%----------------------------------------------------------------------------------------
%	BIBLIOGRAPHIE
%----------------------------------------------------------------------------------------

\addbibresource{bibliographie/bibliographie.bib}

%----------------------------------------------------------------------------------------
%	STYLES
%----------------------------------------------------------------------------------------

\ChNumVar{\fontsize{60}{62}\usefont{OT1}{ptm}{m}{n}\selectfont\textcolor{red}}

\setmainfont[Mapping=tex-text]{Lato}

\geometry{
    paper=a4paper,
    inner=3cm,
    outer=2.5cm,
    top=2.5cm,
    bottom=3.5cm
}

\pagestyle{fancy}
\usepackage{etoolbox}
\patchcmd{\chapter}{\thispagestyle{plain}}{\thispagestyle{fancy}}{}{}
\renewcommand\headrulewidth{0pt}
\fancyhead[L]{}
\fancyhead[C]{}
\fancyhead[R]{}

\makeatletter
\patchcmd{\@makechapterhead}{\vspace*{50\p@}}{\vspace*{-35\p@}}{}{}
\patchcmd{\@makeschapterhead}{\vspace*{50\p@}}{\vspace*{-35\p@}}{}{}
\patchcmd{\DOTI}{\vskip 80\p@}{\vskip 40\p@}{}{}
\patchcmd{\DOTIS}{\vskip 40\p@}{\vskip 0\p@}{}{}
\makeatother

\renewcommand\thesection{\color{red}\thechapter.\arabic{section}}

\newenvironment{acknowledgements} {\renewcommand\abstractname{Remerciements}\begin{abstract}} {\end{abstract}}


\setcounter{tocdepth}{3}
\setcounter{secnumdepth}{3}


%----------------------------------------------------------------------------------------
%	INFORMATIONS
%----------------------------------------------------------------------------------------

\author{Ludwig SIMON}
\title{Mémoire de fin d'études}

\begin{document}

%\subfile{cover/page_garde}
%----------------------------------------------------------------------------------------
%	TITLE PAGE
%----------------------------------------------------------------------------------------

\setlength{\parindent}{0cm}
\setlength{\parskip}{1ex plus 0.5ex minus 0.2ex}
\newcommand{\hsp}{\hspace{20pt}}
\newcommand{\HRule}{\rule{\linewidth}{0.5mm}}

\begin{titlepage}
  \begin{sffamily}
  \begin{center}

    \textsc{\LARGE MASTER MIAGE 2ème année \linebreak Université Paris Nanterre}\\[2cm]

    \textsc{\Large Mémoire de fin d'études}\\[1.5cm]

    \HRule \\[0.4cm]
    { \huge \bfseries Méthodes d'analyse des processus métier pour le choix d'une architecture Big Data adaptée\\[0.4cm] }

    \HRule \\[2cm]
    \includegraphics[scale=0.40]{img/logo_nanterre.jpg}
    \hspace{2cm}
    
    \vfill
  \begin{minipage}{0.4\textwidth}
      \begin{flushleft} \large
        \emph{Auteur :}\\ \textsc{Ludwig SIMON}\\
      \end{flushleft}
    \end{minipage}
    \begin{minipage}{0.4\textwidth}
      \begin{flushright} \large
        \emph{Tuteur :}\\ \textsc{Mcf. Emmanuel HYON}\\
      \end{flushright}
    \end{minipage}
    \vfill
    {\large Février 2019 — Juin 2019}
  \end{center}
  \end{sffamily}
\end{titlepage}

\leavevmode\thispagestyle{empty}\newpage

%----------------------------------------------------------------------------------------
%	Remerciement
%----------------------------------------------------------------------------------------

\begin{acknowledgements}

Je tiens tout d'abord à remercier mon tuteur, monsieur Emmanuel Hyon qui m'a suivi toute l'année pendant la rédaction de mémoire. Les conseils et l'aide qu'il a pu m'apporter sur la rédaction de ce dernier m'ont vraiment été très utiles.

J'aimerais aussi remercier mes collègues d'EDF avec qui j'ai pu travailler dans de bonnes conditions tout au long de l'année. Je les remercie aussi pour le temps qu'ils m'ont accordé afin de répondre à mes interrogations sur le domaine du Big Data pour m'aider dans la rédaction de ce mémoire.

\end{acknowledgements}

\leavevmode\thispagestyle{empty}\newpage

%----------------------------------------------------------------------------------------
%	RESUME
%----------------------------------------------------------------------------------------

\begin{abstract}
	Résumé
\end{abstract}

\leavevmode\thispagestyle{empty}\newpage

%----------------------------------------------------------------------------------------
%	Pre Face
%----------------------------------------------------------------------------------------

\section*{Motivations}
Avant d'avoir commencer mon stage de Master 1 chez EDF dans le service Big Data, pour moi le Big Data était une notion très flou et je ne savais pas vraiment tout ce qui composait ce domaine. Durant ce stage j'ai pu découvrir ce domaine et me rendre compte du nombre conséquent d'outils différents qui sont nécessaire pour la conception d'une architecture Big Data complète. C'est à partir de ce moment là que je me suis demandé, comment est-ce que le choix de l'architecture, et des outils Big Data s'effectue ? Il y a beaucoup plus de possibilités que les domaines que j'ai pu voir jusqu'à maintenant. C'est pour cela que j'ai décidé cette année, d'essayer de comprendre comment fonctionne une architecture Big Data afin de trouver une solution pour choisir plus facilement une architecture Big Data adapté à nos besoins. 

\section*{Objectifs}



Dans ce mémoire, nous allons dans un premier temps rappeler brièvement ce qu'est le Big Data et quand il est vraiment utile de l'utiliser. Dans un second temps, nous allons présenter de manière générale comment est constitué une architecture Big Data. Puis nous verrons plus en détails les architectures qui ont été crées pour répondre aux besoins du Big Data. Ensuite, nous détaillerons pour chaque partie de ces différentes architectures, les solutions logicielles existantes permettant d'accomplir la tâche demandée. Et pour finir, nous allons à partir des études des architectures et des solutions logicielles, essayer de définir un moyen permettant de sélectionner correctement l'architecture et les outils nécessaire pour la création d'une solution Big Data correspondant à nos besoins. Afin de tester que notre méthode de choix est cohérente, on l'appliquera sur une application réel et on la comparera aux autres possibilités d'architecture possible pour s'assurer que c'était le meilleur choix.

\leavevmode\thispagestyle{empty}\newpage

%----------------------------------------------------------------------------------------
%	SOMMAIRE
%----------------------------------------------------------------------------------------

\shorttoc{Sommaire}{1}

%----------------------------------------------------------------------------------------
%	INTRODUCTION
%----------------------------------------------------------------------------------------

\subfile{introduction}

%----------------------------------------------------------------------------------------
%	CHAPITRES
%----------------------------------------------------------------------------------------
\subfile{chapitres/chapitre1}

\subfile{chapitres/chapitre2}

\subfile{chapitres/chapitre3}

\subfile{chapitres/chapitre4}

%----------------------------------------------------------------------------------------
%	CONCLUSION
%----------------------------------------------------------------------------------------

\subfile{conclusion}

%----------------------------------------------------------------------------------------
%	ANNEXES
%----------------------------------------------------------------------------------------

\subfile{annexes/annexes}

%----------------------------------------------------------------------------------------
%	BIBLIOGRAPHIE
%----------------------------------------------------------------------------------------

\printbibliography[heading=bibintoc]
\newpage

%----------------------------------------------------------------------------------------
%	TABLE DES MATIERES
%----------------------------------------------------------------------------------------

\tableofcontents

\listoffigures

\listoftables

\end{document}