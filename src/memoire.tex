\documentclass[12pt, twoside, openright]{report}

%----------------------------------------------------------------------------------------
%	PACKAGES
%----------------------------------------------------------------------------------------

\usepackage{emptypage}
\usepackage{geometry}
\usepackage[utf8x]{inputenc}
\usepackage[french]{babel}
\usepackage[T1]{fontenc}
\usepackage{amsmath}
\usepackage{amsfonts}
\usepackage{amssymb}
\usepackage{graphicx}
\usepackage{subfiles}
\usepackage{fullpage}
\usepackage{fancyhdr}
\usepackage{shorttoc}
\usepackage{fontspec}
\usepackage{xltxtra}
\usepackage{xcolor}
\usepackage{sectsty}
\usepackage[Lenny]{fncychap}
\usepackage[backend=bibtex,sorting=none]{biblatex}

%----------------------------------------------------------------------------------------
%	BIBLIOGRAPHIE
%----------------------------------------------------------------------------------------

\addbibresource{bibliographie/bibliographie.bib}

%----------------------------------------------------------------------------------------
%	STYLES
%----------------------------------------------------------------------------------------

\ChNumVar{\fontsize{60}{62}\usefont{OT1}{ptm}{m}{n}\selectfont\textcolor{red}}

\setmainfont[Mapping=tex-text]{Lato}

\geometry{
    paper=a4paper,
    inner=3cm,
    outer=2.5cm,
    top=2.5cm,
    bottom=3.5cm
}

\pagestyle{fancy}
\usepackage{etoolbox}
\patchcmd{\chapter}{\thispagestyle{plain}}{\thispagestyle{fancy}}{}{}
\renewcommand\headrulewidth{0pt}
\fancyhead[L]{}
\fancyhead[C]{}
\fancyhead[R]{}

\makeatletter
\patchcmd{\@makechapterhead}{\vspace*{50\p@}}{\vspace*{-35\p@}}{}{}
\patchcmd{\@makeschapterhead}{\vspace*{50\p@}}{\vspace*{-35\p@}}{}{}
\patchcmd{\DOTI}{\vskip 80\p@}{\vskip 40\p@}{}{}
\patchcmd{\DOTIS}{\vskip 40\p@}{\vskip 0\p@}{}{}
\makeatother

\renewcommand\thesection{\color{red}\thechapter.\arabic{section}}

\newenvironment{acknowledgements} {\renewcommand\abstractname{Remerciements}\begin{abstract}} {\end{abstract}}


\setcounter{tocdepth}{3}
\setcounter{secnumdepth}{3}


%----------------------------------------------------------------------------------------
%	INFORMATIONS
%----------------------------------------------------------------------------------------

\author{Ludwig SIMON}
\title{Mémoire de fin d'études}

\begin{document}

%\subfile{cover/page_garde}
%----------------------------------------------------------------------------------------
%	TITLE PAGE
%----------------------------------------------------------------------------------------

\setlength{\parindent}{0cm}
\setlength{\parskip}{1ex plus 0.5ex minus 0.2ex}
\newcommand{\hsp}{\hspace{20pt}}
\newcommand{\HRule}{\rule{\linewidth}{0.5mm}}

\begin{titlepage}
  \begin{sffamily}
  \begin{center}

    \textsc{\LARGE MASTER MIAGE 2ème année \linebreak Université Paris Nanterre}\\[2cm]

    \textsc{\Large Mémoire de fin d'études}\\[1.5cm]

    \HRule \\[0.4cm]
    { \huge \bfseries Méthodes d'analyse des processus métier pour le choix d'une architecture Big Data adaptée\\[0.4cm] }

    \HRule \\[2cm]
    \includegraphics[scale=0.40]{img/logo_nanterre.jpg}
    \hspace{2cm}
    
    \vfill
  \begin{minipage}{0.4\textwidth}
      \begin{flushleft} \large
        \emph{Auteur :}\\ \textsc{Ludwig SIMON}\\
      \end{flushleft}
    \end{minipage}
    \begin{minipage}{0.4\textwidth}
      \begin{flushright} \large
        \emph{Tuteur :}\\ \textsc{Mcf. Emmanuel HYON}\\
      \end{flushright}
    \end{minipage}
    \vfill
    {\large Février 2019 — Juin 2019}
  \end{center}
  \end{sffamily}
\end{titlepage}

\leavevmode\thispagestyle{empty}\newpage

%----------------------------------------------------------------------------------------
%	Remerciement
%----------------------------------------------------------------------------------------

\begin{acknowledgements}

Remerciements

\end{acknowledgements}

\leavevmode\thispagestyle{empty}\newpage

%----------------------------------------------------------------------------------------
%	RESUME
%----------------------------------------------------------------------------------------

\begin{abstract}
	Résumé
\end{abstract}

\leavevmode\thispagestyle{empty}\newpage

%----------------------------------------------------------------------------------------
%	Pre Face
%----------------------------------------------------------------------------------------

\section*{Motivations}
Le Big Data est un domaine très vaste, il y a une multitude d'outils pour effectuer les "mêmes" tâches. Il est donc très difficile de faire le bon choix lorsqu'on se lance dans ce domaine.

\section*{Objectifs}

Dans ce mémoire, nous allons dans un premier temps rappeler brièvement ce qu'est le Big Data et quand il est vraiment utile de l'utiliser. Dans un second temps nous allons devoir établir des critères permettant d'évaluer quelle catégorie d'outil convient le mieux et par la suite quel outil conviendra le mieux. Pour le choix de l'outil on utilisera en plus de critères de cas d'utilisation des benchmarks afin de récupérer des informations sur la consommation de ressources et de temps pour chaque outil. Ensuite, à l'aide de ces critères définit précédemment, nous allons définir des méthodes d'analyse afin de sélectionner les outils adéquat. Et pour finir nous allons appliquer nos méthodes sur un cas concret afin de vérifier l'efficacité de notre solution.

\leavevmode\thispagestyle{empty}\newpage

%----------------------------------------------------------------------------------------
%	SOMMAIRE
%----------------------------------------------------------------------------------------

\shorttoc{Sommaire}{1}

%----------------------------------------------------------------------------------------
%	INTRODUCTION
%----------------------------------------------------------------------------------------

\subfile{introduction}

%----------------------------------------------------------------------------------------
%	CHAPITRES
%----------------------------------------------------------------------------------------
\subfile{chapitres/chapitre1}

\subfile{chapitres/chapitre2}

\subfile{chapitres/chapitre3}

\subfile{chapitres/chapitre4}

%----------------------------------------------------------------------------------------
%	ANNEXES
%----------------------------------------------------------------------------------------

\subfile{annexes/annexes}

%----------------------------------------------------------------------------------------
%	BIBLIOGRAPHIE
%----------------------------------------------------------------------------------------

\printbibliography[heading=bibintoc]
\newpage

%----------------------------------------------------------------------------------------
%	TABLE DES MATIERES
%----------------------------------------------------------------------------------------

\tableofcontents

\listoffigures

\end{document}