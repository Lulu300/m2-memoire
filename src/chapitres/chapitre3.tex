\documentclass[memoire.tex]{subfiles}

\chapter{Critères d'analyse}

Maintenant que nous sommes familier avec les architectures Big Data ainsi qu'avec les différentes solutions logicielle, nous allons pouvoir définir des critères afin de sélectionner l'architecture qui correspond à notre besoin. Certains critères ont déjà été dégagé au fil de ce mémoire. La première étape va être de définir des critères par rapport au choix de l'architecture et ensuite des critères pour choisir les solution logicielles à utiliser au sein de cette architecture. En plus des différentes informations que nous avons exposé jusqu'ici, nous aurons besoin de prendre en compte le cas d'utilisation nécessitant le déploiement d'une architecture Big Data. 

\section{Critères pour le choix de l'architecture}

Avant de voir les différent critères par rapport au choix de l'architecture, je tiens à rappeler ce qui a été expliquer en introduction de ce mémoire, c'est qu'avant de passer sur une architecture Big Data il faut s'assurer d'en avoir l'utilité.



\section{Type de traitement des données}
Streaming - Micro Batch - Batch

\section{Format des données}

\section{Perte de données admissible}

\section{Volumétrie}

\section{Performance}